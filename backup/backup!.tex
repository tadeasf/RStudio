% Options for packages loaded elsewhere
\PassOptionsToPackage{unicode}{hyperref}
\PassOptionsToPackage{hyphens}{url}
%
\documentclass[
]{article}
\usepackage{amsmath,amssymb}
\usepackage{lmodern}
\usepackage{ifxetex,ifluatex}
\ifnum 0\ifxetex 1\fi\ifluatex 1\fi=0 % if pdftex
  \usepackage[T1]{fontenc}
  \usepackage[utf8]{inputenc}
  \usepackage{textcomp} % provide euro and other symbols
\else % if luatex or xetex
  \usepackage{unicode-math}
  \defaultfontfeatures{Scale=MatchLowercase}
  \defaultfontfeatures[\rmfamily]{Ligatures=TeX,Scale=1}
\fi
% Use upquote if available, for straight quotes in verbatim environments
\IfFileExists{upquote.sty}{\usepackage{upquote}}{}
\IfFileExists{microtype.sty}{% use microtype if available
  \usepackage[]{microtype}
  \UseMicrotypeSet[protrusion]{basicmath} % disable protrusion for tt fonts
}{}
\makeatletter
\@ifundefined{KOMAClassName}{% if non-KOMA class
  \IfFileExists{parskip.sty}{%
    \usepackage{parskip}
  }{% else
    \setlength{\parindent}{0pt}
    \setlength{\parskip}{6pt plus 2pt minus 1pt}}
}{% if KOMA class
  \KOMAoptions{parskip=half}}
\makeatother
\usepackage{xcolor}
\IfFileExists{xurl.sty}{\usepackage{xurl}}{} % add URL line breaks if available
\IfFileExists{bookmark.sty}{\usepackage{bookmark}}{\usepackage{hyperref}}
\hypersetup{
  pdftitle={Úvod do kvantitativní textové analýzy},
  pdfauthor={Tadeášek},
  hidelinks,
  pdfcreator={LaTeX via pandoc}}
\urlstyle{same} % disable monospaced font for URLs
\usepackage[margin=1in]{geometry}
\usepackage{color}
\usepackage{fancyvrb}
\newcommand{\VerbBar}{|}
\newcommand{\VERB}{\Verb[commandchars=\\\{\}]}
\DefineVerbatimEnvironment{Highlighting}{Verbatim}{commandchars=\\\{\}}
% Add ',fontsize=\small' for more characters per line
\usepackage{framed}
\definecolor{shadecolor}{RGB}{248,248,248}
\newenvironment{Shaded}{\begin{snugshade}}{\end{snugshade}}
\newcommand{\AlertTok}[1]{\textcolor[rgb]{0.94,0.16,0.16}{#1}}
\newcommand{\AnnotationTok}[1]{\textcolor[rgb]{0.56,0.35,0.01}{\textbf{\textit{#1}}}}
\newcommand{\AttributeTok}[1]{\textcolor[rgb]{0.77,0.63,0.00}{#1}}
\newcommand{\BaseNTok}[1]{\textcolor[rgb]{0.00,0.00,0.81}{#1}}
\newcommand{\BuiltInTok}[1]{#1}
\newcommand{\CharTok}[1]{\textcolor[rgb]{0.31,0.60,0.02}{#1}}
\newcommand{\CommentTok}[1]{\textcolor[rgb]{0.56,0.35,0.01}{\textit{#1}}}
\newcommand{\CommentVarTok}[1]{\textcolor[rgb]{0.56,0.35,0.01}{\textbf{\textit{#1}}}}
\newcommand{\ConstantTok}[1]{\textcolor[rgb]{0.00,0.00,0.00}{#1}}
\newcommand{\ControlFlowTok}[1]{\textcolor[rgb]{0.13,0.29,0.53}{\textbf{#1}}}
\newcommand{\DataTypeTok}[1]{\textcolor[rgb]{0.13,0.29,0.53}{#1}}
\newcommand{\DecValTok}[1]{\textcolor[rgb]{0.00,0.00,0.81}{#1}}
\newcommand{\DocumentationTok}[1]{\textcolor[rgb]{0.56,0.35,0.01}{\textbf{\textit{#1}}}}
\newcommand{\ErrorTok}[1]{\textcolor[rgb]{0.64,0.00,0.00}{\textbf{#1}}}
\newcommand{\ExtensionTok}[1]{#1}
\newcommand{\FloatTok}[1]{\textcolor[rgb]{0.00,0.00,0.81}{#1}}
\newcommand{\FunctionTok}[1]{\textcolor[rgb]{0.00,0.00,0.00}{#1}}
\newcommand{\ImportTok}[1]{#1}
\newcommand{\InformationTok}[1]{\textcolor[rgb]{0.56,0.35,0.01}{\textbf{\textit{#1}}}}
\newcommand{\KeywordTok}[1]{\textcolor[rgb]{0.13,0.29,0.53}{\textbf{#1}}}
\newcommand{\NormalTok}[1]{#1}
\newcommand{\OperatorTok}[1]{\textcolor[rgb]{0.81,0.36,0.00}{\textbf{#1}}}
\newcommand{\OtherTok}[1]{\textcolor[rgb]{0.56,0.35,0.01}{#1}}
\newcommand{\PreprocessorTok}[1]{\textcolor[rgb]{0.56,0.35,0.01}{\textit{#1}}}
\newcommand{\RegionMarkerTok}[1]{#1}
\newcommand{\SpecialCharTok}[1]{\textcolor[rgb]{0.00,0.00,0.00}{#1}}
\newcommand{\SpecialStringTok}[1]{\textcolor[rgb]{0.31,0.60,0.02}{#1}}
\newcommand{\StringTok}[1]{\textcolor[rgb]{0.31,0.60,0.02}{#1}}
\newcommand{\VariableTok}[1]{\textcolor[rgb]{0.00,0.00,0.00}{#1}}
\newcommand{\VerbatimStringTok}[1]{\textcolor[rgb]{0.31,0.60,0.02}{#1}}
\newcommand{\WarningTok}[1]{\textcolor[rgb]{0.56,0.35,0.01}{\textbf{\textit{#1}}}}
\usepackage{graphicx}
\makeatletter
\def\maxwidth{\ifdim\Gin@nat@width>\linewidth\linewidth\else\Gin@nat@width\fi}
\def\maxheight{\ifdim\Gin@nat@height>\textheight\textheight\else\Gin@nat@height\fi}
\makeatother
% Scale images if necessary, so that they will not overflow the page
% margins by default, and it is still possible to overwrite the defaults
% using explicit options in \includegraphics[width, height, ...]{}
\setkeys{Gin}{width=\maxwidth,height=\maxheight,keepaspectratio}
% Set default figure placement to htbp
\makeatletter
\def\fps@figure{htbp}
\makeatother
\setlength{\emergencystretch}{3em} % prevent overfull lines
\providecommand{\tightlist}{%
  \setlength{\itemsep}{0pt}\setlength{\parskip}{0pt}}
\setcounter{secnumdepth}{-\maxdimen} % remove section numbering
\ifluatex
  \usepackage{selnolig}  % disable illegal ligatures
\fi

\title{Úvod do kvantitativní textové analýzy}
\author{Tadeášek}
\date{12/7/2021}

\begin{document}
\maketitle

\hypertarget{nejlepux161ejux161uxed-nuxe1vod-na-vstup-do-matrixu}{%
\section{\texorpdfstring{\emph{Nejlepšejší návod na vstup do
Matrixu}}{Nejlepšejší návod na vstup do Matrixu}}\label{nejlepux161ejux161uxed-nuxe1vod-na-vstup-do-matrixu}}

Připrav si ručník, nepanikař, poděkuj rybám a jdem na to!

\hypertarget{pux159uxedprava-rka}{%
\subsection{Příprava Rka}\label{pux159uxedprava-rka}}

Nejdřív je důležitý se na zákrok připravit, proto si nainstaluj tyhle
zcela balíčky skrze příkaz: install.packages(``balíček'')

``SnowballC'' -\textgreater{} klíčový pro ``text stemming''\\
``wordcloud'' -\textgreater{} to je docela očividný, viď! vono to nebude
tak těžký nakonec.\\
``RColorBrewer'' -\textgreater{} pro všechny visual driven jedince
klíčovej balíček aneb vymaluj si svůj mráček ``stopwords''
-\textgreater{} databáze stop slov v různejch jazycích + dobrých stop
slov pro stemming\\
``quanteda'' -\textgreater{} balíček, který využijeme pro odstranění
bezvýznamnových slov - proč?:
\url{https://stackoverflow.com/questions/26899857/self-conflicting-stopwords-in-r-tm-text-mining}
+ další skvělá dokumentace zde:
\url{https://quanteda.io/articles/pkgdown/quickstart.html\#creating-a-corpus}
``readtext'' -\textgreater{} čtení slov z .txt\\
``quanteda.textplots'' -\textgreater{} rozšíření quantedy pro tvorbu
wordcloudů\\
``quanteda.textstats'' -\textgreater{} rozšíření quantedy pro tvorbu
grafů z textu\\
``ggplot2'' -\textgreater{} snad ani nemusím představovat:-)

Po instalaci všechny balíčky zapni skrz příkaz: library(``balíček'').
Pro milovníky oken se dají balíčky hledat a zapínat vpravo dole:
\url{https://i.imgur.com/8Z3MsGA.png}

\begin{Shaded}
\begin{Highlighting}[]
\FunctionTok{library}\NormalTok{(}\StringTok{"quanteda"}\NormalTok{)}
\end{Highlighting}
\end{Shaded}

\begin{verbatim}
## Package version: 3.2.0
## Unicode version: 13.0
## ICU version: 69.1
\end{verbatim}

\begin{verbatim}
## Parallel computing: 12 of 12 threads used.
\end{verbatim}

\begin{verbatim}
## See https://quanteda.io for tutorials and examples.
\end{verbatim}

\begin{Shaded}
\begin{Highlighting}[]
\FunctionTok{library}\NormalTok{(}\StringTok{"readtext"}\NormalTok{)}
\FunctionTok{library}\NormalTok{(}\StringTok{"stopwords"}\NormalTok{)}
\FunctionTok{library}\NormalTok{(}\StringTok{"SnowballC"}\NormalTok{)}
\FunctionTok{library}\NormalTok{(}\StringTok{"wordcloud"}\NormalTok{)}
\end{Highlighting}
\end{Shaded}

\begin{verbatim}
## Loading required package: RColorBrewer
\end{verbatim}

\begin{Shaded}
\begin{Highlighting}[]
\FunctionTok{library}\NormalTok{(}\StringTok{"RColorBrewer"}\NormalTok{)}
\FunctionTok{library}\NormalTok{(}\StringTok{"quanteda.textplots"}\NormalTok{)}
\FunctionTok{library}\NormalTok{(}\StringTok{"quanteda.textstats"}\NormalTok{)}
\FunctionTok{library}\NormalTok{(}\StringTok{"ggplot2"}\NormalTok{)}
\end{Highlighting}
\end{Shaded}

\hypertarget{pux159uxedprava-textu-k-mruxe1ux10dkovuxe1nuxed}{%
\subsection{Příprava textu k
mráčkování}\label{pux159uxedprava-textu-k-mruxe1ux10dkovuxe1nuxed}}

Teď je ten moment, kdy chcem zjistit, co chceme zpracovávat. Já třeba
žiju rád nebezpečně, takže jsem šel na infamous stránku Library Genesis
(\url{https://libgen.is/}). Rozhodl jsem se zpracovat knihu z mého
oblíbeného žánru magickýho realismu - Konec světa\&Hardboiled Wonderland
od Murakamiho.

Většina knih je na Libgenu ve formátu .epub. Já zvolil nejjednodušší
cestu a jako vstup do celého procesu textového těžení jsem si zvolil
.txt. Konverze je poměrně jednoduchá, využít se dá například tato
služba: \url{https://convertio.co/epub-txt/}. Netřeba za tím hledat
nástrahy, jde o přímočarý proces.

\hypertarget{koneux10dnux11b-kuxf3dujem-huruxe1}{%
\subsection{Konečně kódujem,
hurá}\label{koneux10dnux11b-kuxf3dujem-huruxe1}}

Textovej soubor s knížkou teda máme. Ten chceme naloadovat. Tady už
zjistíme, že se můj kód mění od referečního, který používá knihovnu
textmining (tm). Po intenzivním googlení jsem usoudil, že nejlepším
nástrojem na kvantitativní textovou analýzu je quanteda. Ta má trochu
jiné nástroje než tm, ale zas mi to pomohlo pochopit fungování Rka o
něco lépe.

Nejdřív načteme slova z našeho textového souboru do datasetu:

\begin{Shaded}
\begin{Highlighting}[]
\NormalTok{KonecSvěta }\OtherTok{\textless{}{-}} \FunctionTok{readtext}\NormalTok{(}\StringTok{"\textasciitilde{}/GitHub/RStudio/data murakami/Murakami.txt"}\NormalTok{)}
\end{Highlighting}
\end{Shaded}

Pokud nevíš, jak jednoduše vykopírovat cestu k souboru, tak je to tady:
\includegraphics{copypath.png}

Teď budeme potřebovat Corpus. Výhoda quantedy je třeba v tom, že data v
Corpusu při čištění nemění - využívá místo toho tokeny. Token se vytvoří
takto:

\begin{Shaded}
\begin{Highlighting}[]
\NormalTok{KorpusWonderland }\OtherTok{\textless{}{-}} \FunctionTok{corpus}\NormalTok{(KonecSveta)}
\FunctionTok{summary}\NormalTok{(KorpusWonderland)}
\end{Highlighting}
\end{Shaded}

\begin{verbatim}
## Corpus consisting of 1 document, showing 1 document:
## 
##          Text Types Tokens Sentences
##  Murakami.txt 12571 144076     11856
\end{verbatim}

Je vidět, že Corpus je teď zmatenej, potřebujeme mu říct, co v textu je,
z čeho má tokeny vytvářet - obecně v Rku používáme vektory, který nám
říkaj, co daná věc znamená - mohou to být čísla, data, jména, roky! V
tomhle případě nám ale bude stačit jednoduchý charakterový vektor. - to
je první část kódu

V druhé části si připravíme Tokeny, které potom můžeme dále čistit.

\begin{Shaded}
\begin{Highlighting}[]
\FunctionTok{as.character}\NormalTok{(KorpusWonderland)[}\DecValTok{2}\NormalTok{]}
\end{Highlighting}
\end{Shaded}

\begin{verbatim}
## <NA> 
##   NA
\end{verbatim}

\begin{Shaded}
\begin{Highlighting}[]
\FunctionTok{summary}\NormalTok{(KorpusWonderland, }\AttributeTok{n =} \DecValTok{1}\NormalTok{)}
\end{Highlighting}
\end{Shaded}

\begin{verbatim}
## Corpus consisting of 1 document, showing 1 document:
## 
##          Text Types Tokens Sentences
##  Murakami.txt 12571 144076     11856
\end{verbatim}

Nyní připravíme tokeny v samotném korpusu. Jde použít jen neinvazivní
příkazy - použil jsem naprostou většinu z nich, více v dokumentaci o
quantedě.

\begin{Shaded}
\begin{Highlighting}[]
\FunctionTok{tokens}\NormalTok{(KorpusWonderland, }\AttributeTok{remove\_numbers =} \ConstantTok{TRUE}\NormalTok{, }\AttributeTok{remove\_punct =} \ConstantTok{TRUE}\NormalTok{, }\AttributeTok{remove\_symbols =} \ConstantTok{TRUE}\NormalTok{, }\AttributeTok{remove\_url =} \ConstantTok{TRUE}\NormalTok{, }\AttributeTok{remove\_separators =} \ConstantTok{TRUE}\NormalTok{)}
\end{Highlighting}
\end{Shaded}

\begin{verbatim}
## Tokens consisting of 1 document.
## Murakami.txt :
##  [1] "HARD-BOILED" "WONDERLAND"  "AND"         "THE"         "END"        
##  [6] "OF"          "THE"         "WORLD"       "by"          "Haruki"     
## [11] "Murakami"    "TRANSLATED" 
## [ ... and 116,762 more ]
\end{verbatim}

Teď si vytvořím separátní set tokenů - v tom je zase quanteda lepší než
tm - TM zasahuje při čištění přímo do korpusu, kdežto v quantedě
předcházíme případné ztrátě dat a progresu již takto. Následujícím
příkazem vytváříme data set Tokeny, který obsahuje tokeny z našeho
korpusu.

\begin{Shaded}
\begin{Highlighting}[]
\NormalTok{Tokeny }\OtherTok{\textless{}{-}} \FunctionTok{tokens}\NormalTok{(KorpusWonderland, }\AttributeTok{remove\_numbers =} \ConstantTok{TRUE}\NormalTok{, }\AttributeTok{remove\_punct =} \ConstantTok{TRUE}\NormalTok{, }\AttributeTok{remove\_symbols =} \ConstantTok{TRUE}\NormalTok{, }\AttributeTok{remove\_url =} \ConstantTok{TRUE}\NormalTok{, }\AttributeTok{remove\_separators =} \ConstantTok{TRUE}\NormalTok{)}
\end{Highlighting}
\end{Shaded}

V dalším kroku chci, aby bylo všechno v lowercase, na to je tento příkaz
(tady používám dokumentaci, nic složitého)

\begin{Shaded}
\begin{Highlighting}[]
\NormalTok{MaleTokeny }\OtherTok{\textless{}{-}} \FunctionTok{tokens\_tolower}\NormalTok{(Tokeny, }\AttributeTok{keep\_acronyms =} \ConstantTok{FALSE}\NormalTok{)}
\end{Highlighting}
\end{Shaded}

No a ted prichazi zajimavejsi cast. Chci se zbavit bezvyznamovych slow
(a, the,\ldots). Na to vyuziju balicek stopwords. Opět, balíček tm
využívat externí stopwords neumí, už ten příkaz v sobě má a je dost na
prd. V quantedě můžeme využít snowball stopwords databázi.

příkaz je následující - vyberu z malých tokenů ty, které odpovídají
databázi stopwords od snowball a pak řeknu, že jejich označení znamená,
že je odstraním

\begin{Shaded}
\begin{Highlighting}[]
\NormalTok{CisteTokeny }\OtherTok{\textless{}{-}} \FunctionTok{tokens\_select}\NormalTok{(MaleTokeny, }\AttributeTok{pattern =} \FunctionTok{stopwords}\NormalTok{(}\AttributeTok{language =} \StringTok{"en"}\NormalTok{, }\AttributeTok{source =} \StringTok{"snowball"}\NormalTok{, }\AttributeTok{simplify =} \ConstantTok{TRUE}\NormalTok{), }\AttributeTok{selection =} \StringTok{"remove"}\NormalTok{)}
\FunctionTok{print}\NormalTok{(CisteTokeny)}
\end{Highlighting}
\end{Shaded}

\begin{verbatim}
## Tokens consisting of 1 document.
## Murakami.txt :
##  [1] "hard-boiled" "wonderland"  "end"         "world"       "haruki"     
##  [6] "murakami"    "translated"  "japanese"    "alfred"      "birnbaum"   
## [11] "haruki"      "murakami"   
## [ ... and 57,085 more ]
\end{verbatim}

Dostala ses až sem? Tak trocha egoboostu, tohle píšou v dokumentaci ke
quantedě ``Tokenizing texts is an intermediate option, and most users
will want to skip straight to constructing a document-feature matrix''

Pokračujeme ale dál. Nyní si chci vybrat další slova, která se mi v
CisteTokeny nelíbí. To udělám skrze arguemnt pattern:

\begin{Shaded}
\begin{Highlighting}[]
\NormalTok{ManualniNiceni }\OtherTok{\textless{}{-}} \FunctionTok{tokens\_select}\NormalTok{(CisteTokeny, }\AttributeTok{pattern =} \FunctionTok{c}\NormalTok{(}\StringTok{"said"}\NormalTok{, }\StringTok{"like"}\NormalTok{, }\StringTok{"one"}\NormalTok{, }\StringTok{"say"}\NormalTok{, }\StringTok{"can"}\NormalTok{, }\StringTok{"go"}\NormalTok{, }\StringTok{"right"}\NormalTok{), }\AttributeTok{selection =} \StringTok{"remove"}\NormalTok{)}
\end{Highlighting}
\end{Shaded}

\hypertarget{matrix}{%
\subsection{Matrix?}\label{matrix}}

Teď mám tedy vyčištěné tokeny a připravený korpus, vzhůru do Matrixu!

Na to je příkaz následující, pak si jen vytisknem pár znaků, abychom se
ujistili, že to je správně:

\begin{Shaded}
\begin{Highlighting}[]
\NormalTok{MatrixWonderland }\OtherTok{\textless{}{-}} \FunctionTok{tokens}\NormalTok{(ManualniNiceni) }\SpecialCharTok{\%\textgreater{}\%}
  \FunctionTok{dfm}\NormalTok{()}
\NormalTok{MatrixWonderland[, }\DecValTok{1}\SpecialCharTok{:}\DecValTok{30}\NormalTok{]}
\end{Highlighting}
\end{Shaded}

\begin{verbatim}
## Document-feature matrix of: 1 document, 30 features (0.00% sparse) and 0 docvars.
##               features
## docs           hard-boiled wonderland end world haruki murakami translated
##   Murakami.txt           1          2 104   196      3        3          3
##               features
## docs           japanese alfred birnbaum
##   Murakami.txt        9      2        2
## [ reached max_nfeat ... 20 more features ]
\end{verbatim}

Šikovnou funkcí je wordstem - stemming - odstraníme přebytečná písmenka
a získáme tím jen kořeny slov:

\begin{Shaded}
\begin{Highlighting}[]
\NormalTok{MatrixWonderland }\OtherTok{\textless{}{-}} \FunctionTok{dfm\_wordstem}\NormalTok{(MatrixWonderland, }\AttributeTok{language =}\NormalTok{ (}\StringTok{"english"}\NormalTok{))}
\end{Highlighting}
\end{Shaded}

Teď se podíváme na naše top slova:

\begin{Shaded}
\begin{Highlighting}[]
\FunctionTok{topfeatures}\NormalTok{(MatrixWonderland, }\DecValTok{80}\NormalTok{)}
\end{Highlighting}
\end{Shaded}

\begin{verbatim}
##      time      know     thing       old       get      look    shadow      back 
##       348       321       271       255       249       248       238       231 
##       now      hand      just     world       ask     skull      come      even 
##       218       211       203       201       198       197       196       189 
##     think      wall       way       got      mind      take       eye     light 
##       186       184       182       177       175       172       172       171 
##      head      want      much    someth       say       see      well       man 
##       165       164       163       163       161       160       159       158 
##      make       two     sound      noth      town     still  gatekeep      tell 
##       158       153       153       149       149       146       142       140 
##      open      long      door     never    around      seem     place        go 
##       138       138       136       133       132       131       126       125 
##       day      work        us      read       end grandfath     first   thought 
##       124       123       122       120       118       117       114       114 
##      left      turn      good      went      girl      must    system professor 
##       113       113       112       111       111       110       110       109 
##      wood      leav    realli      need     littl     water     three      away 
##       105       104       104       104       102       102       101       101 
##      next       put     anyth      mean     anoth   everyth      live      feel 
##       100        99        99        98        97        96        96        94
\end{verbatim}

\hypertarget{vizualizace-mruxe1ux10dky-a-grafuxedky}{%
\subsection{Vizualizace: mráčky a
grafíky}\label{vizualizace-mruxe1ux10dky-a-grafuxedky}}

Tak! A je to tady, teď už jdeme dělat mráček. Nastavíme si seed, aby se
nám generoval stále stejně, když ho budeme chtít upravovat. Argumenty
jsou docela jasné, v dokumentaci dobře popsané. Využíváme tu na barvení
knihovnu RColorBrewer. Abych neměl stejný graf jako všichni, tak jsem
využil sekvenční paletu s co nejvyšším množstvím odlišení, v tomto
případě je barev 9.

\begin{Shaded}
\begin{Highlighting}[]
\FunctionTok{set.seed}\NormalTok{(}\DecValTok{100}\NormalTok{)}
\FunctionTok{textplot\_wordcloud}\NormalTok{(MatrixWonderland, }\AttributeTok{min\_count =} \DecValTok{100}\NormalTok{, }\AttributeTok{random\_order =} \ConstantTok{FALSE}\NormalTok{, }\AttributeTok{rotation =} \FloatTok{0.25}\NormalTok{, }\AttributeTok{color =}\NormalTok{ RColorBrewer}\SpecialCharTok{::}\FunctionTok{brewer.pal}\NormalTok{(}\DecValTok{9}\NormalTok{, }\StringTok{"YlGnBu"}\NormalTok{))}
\end{Highlighting}
\end{Shaded}

\includegraphics{backup!_files/figure-latex/unnamed-chunk-13-1.pdf}

Pak jsem chtěl udělat ještě grafík. Dokumentace tady:
\url{https://quanteda.io/articles/pkgdown/examples/plotting.html} Jestli
ses dostal/a až sem, tak už to vůbec není těžké pochopit:-)

\begin{Shaded}
\begin{Highlighting}[]
\NormalTok{FrequencyPlot }\OtherTok{\textless{}{-}} \FunctionTok{textstat\_frequency}\NormalTok{(MatrixWonderland, }\AttributeTok{n =} \DecValTok{100}\NormalTok{)}
\NormalTok{FrequencyPlot}\SpecialCharTok{$}\NormalTok{feature }\OtherTok{\textless{}{-}} \FunctionTok{with}\NormalTok{(FrequencyPlot, }\FunctionTok{reorder}\NormalTok{(feature, }\SpecialCharTok{{-}}\NormalTok{frequency))}
\end{Highlighting}
\end{Shaded}

\begin{Shaded}
\begin{Highlighting}[]
\FunctionTok{ggplot}\NormalTok{(FrequencyPlot, }\FunctionTok{aes}\NormalTok{(}\AttributeTok{x =}\NormalTok{ feature, }\AttributeTok{y =}\NormalTok{ frequency)) }\SpecialCharTok{+}
  \FunctionTok{geom\_point}\NormalTok{() }\SpecialCharTok{+}
  \FunctionTok{theme}\NormalTok{(}\AttributeTok{axis.text.x =} \FunctionTok{element\_text}\NormalTok{(}\AttributeTok{angle =} \DecValTok{90}\NormalTok{, }\AttributeTok{hjust =} \DecValTok{1}\NormalTok{))}
\end{Highlighting}
\end{Shaded}

\includegraphics{backup!_files/figure-latex/unnamed-chunk-15-1.pdf}

\hypertarget{odkazy-na-zdroje}{%
\section{Odkazy na zdroje}\label{odkazy-na-zdroje}}

Neseřazeno, jak jsem to pomíchal, tak to sem posílám

\url{https://quanteda.io/articles/pkgdown/examples/plotting.html}~\\
\url{https://stackoverflow.com/questions/47039236/how-to-keep-wordcloud-layout-in-r}~\\
\url{https://stackoverflow.com/questions/26899857/self-conflicting-stopwords-in-r-tm-text-mining}
-\textgreater{} proto quanteda, tm sucks!\\
\url{https://quanteda.io/articles/pkgdown/comparison.html}
-\textgreater{} proto quanteda, je prostě dobrá! na kvantitativní
analýzu textu není nic lepšího\\
\url{https://quanteda.io/reference/tokens.html} -\textgreater{} Jak
používat tokeny\\
\url{https://tutorials.quanteda.io/basic-operations/tokens/tokens_select/}
-\textgreater{} jak vybrat a odstranit tokeny, který nechcem\\
\url{https://quanteda.io/reference/stopwords.html} -\textgreater{}
stopwords

\url{https://quanteda.io/reference/tokens_tolower.html} -\textgreater{}
jak hodit vše do lowercase

\url{https://quanteda.io/articles/quickstart.html} -\textgreater{}
základní návod na to, jak nahrát .txt, jak ho přečíst, jak ho převést do
corpusu, jak pracovat s tokenama a jak udělat wordcloud případně
vizualizovat ggplotem

\url{https://quanteda.io/articles/pkgdown/examples/plotting.html}
-\textgreater{} širší možnosti vizualizace

\url{https://quanteda.io/reference/textplot_wordcloud.html}
-\textgreater{} values pro wordcloud v quantedě

\url{https://www.datanovia.com/en/blog/the-a-z-of-rcolorbrewer-palette/\#display-all-brewer-palettes}
-\textgreater{} rcolorbrewer palety

\url{https://quanteda.io/articles/pkgdown/replication/digital-humanities.html}
-\textgreater{} digital humanities porno! nástroje, který nám ukazoval
Josef jsou skvělý v tom, jak jsou jednoduchý, ale tohle toho umí tolik!
TOLIK!

Palety textově: Sequential palettes (first list of colors), which are
suited to ordered data that progress from low to high (gradient). The
palettes names are : Blues, BuGn, BuPu, GnBu, Greens, Greys, Oranges,
OrRd, PuBu, PuBuGn, PuRd, Purples, RdPu, Reds, YlGn, YlGnBu YlOrBr,
YlOrRd.\\
Qualitative palettes (second list of colors), which are best suited to
represent nominal or categorical data. They not imply magnitude
differences between groups. The palettes names are : Accent, Dark2,
Paired, Pastel1, Pastel2, Set1, Set2, Set3.\\
Diverging palettes (third list of colors), which put equal emphasis on
mid-range critical values and extremes at both ends of the data range.
The diverging palettes are : BrBG, PiYG, PRGn, PuOr, RdBu, RdGy, RdYlBu,
RdYlGn, Spectral

\hypertarget{tipy-na-konec}{%
\subsection{Tipy na konec}\label{tipy-na-konec}}

instalace externích repositories se provádí skrz balíček ``remotes''

\end{document}
